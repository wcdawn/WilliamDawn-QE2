\section{Conclusions and Implications}
\label{sec:conclusion}

  In \citetitle{qe2paper}, \citeauthor{qe2paper} conclude that the proposed
  method of solving the \gls{nem} equations with the \gls{jfnk} method after
  performing local elimination and using the physics based preconditioner
  provides improved convergence rate and reduced computation time compared to
  their presentation of the \gls{pi} method. It is difficult to argue with these
  results as presented. However, \sref{sec:critique} has shown that it is the
  data omitted from the results that merits consideration. Otherwise, it is
  difficult to extend the results of \citeauthor{qe2paper} to broader
  applications of the \gls{jfnk} method for solving the \gls{nem} equations.

  The first step for better interpreting the results presented is to compare the
  \gls{jfnk} method to a \gls{pi} method using the \gls{ws} as in the work by
  \citeauthor{jfnk_wielandt}. It would also be useful to investigate
  preconditioning with a few \glspl{pi} as this has been demonstrated to be the
  preferable preconditioning by others \cite{gill_azmy,jfnk_wielandt}. Finally,
  the \gls{nem} should be implemented in the $\dtilde$ formulation to improve
  computational efficiency and allow for compatibility with existing
  production-quality computer programs \cite{palmtagThesis,smith_nonlinear}.

  In their concluding remarks, \citeauthor{qe2paper} claim that their method
  will be extended to ``improve the computational efficiency for large-scale
  complicated multiphysics coupled problems in nuclear reactor analysis''
  \cite{qe2paper}. This may seem like a straightforward extension, but a new
  residual function and Jacobian approximation will likely be required.

  The authors' claim is extremely bold given both the lack of important results
  and other publications in the field. Specifically, \citeauthor{caslJFNK}
  investigated this exact application of \gls{jfnk} to large-scale multiphysics
  simulations. Such an application required significant work in the formation of
  the Jacobian including constructing approximations of temperature derivatives
  of cross sections. These cross section derivatives were specific to a
  particular type of reactor (\glspl{pwr}) but \citeauthor{qe2paper} attempt to
  simulate both \glspl{pwr} and pebble bed reactor designs. Succinctly, in
  full reactor multiphysics simulations, it was determined that without the
  calculation of cross section derivatives, the \gls{jfnk} method was
  unacceptable as it required a burdensome amount of cross section processing 
  time during the simulation \cite{caslJFNK}.

  The results of \citeauthor{caslJFNK} are not the end of the narrative. 
  Applying the \gls{jfnk} method to the \gls{nem} equations may be useful for
  certain nuclear reactor multiphysics simulations. However, insufficient data 
  is provided by \citeauthor{qe2paper} to make such a conclusion. Additional
  attention must be paid in the application of the \gls{jfnk} method to the
  \gls{nem} equations to determine if such an implementation would provide
  a benefit compared to existing methods. When drawing conclusions about
  improvements in computational efficiency due to a new method, it is crucial to 
  compare optimized codes to determine if any improvement is true, or merely due
  to a suboptimal implementation.
