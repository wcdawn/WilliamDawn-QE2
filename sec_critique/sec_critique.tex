\section{Critique}
\label{sec:critique}
  The primary motivation for this paper is to critique the article by
  \citeauthor{qe2paper}. While there may be more to critique, the discussion
  here focuses on the application of proposed the \gls{jfnk} to realistic
  reactor simulation problems.

  \subsection{Verification and Validation}
    Had the authors provided additional results, the implications of the
    publication may be more significant. However, important data has not been
    reported by the authors. To demonstrate the implementation of their method,
    the authors compare their converged eigenvalue and scalar flux distribution 
    to a single benchmark problem: the \gls{iaea} 3D \gls{pwr} problem. While
    results are presented for a second pebble bed reactor, these results have
    not been previously verified so the authors verify against a similar version
    of the same code. The authors do not demonstrate the spatial convergence
    order of their method. Typically spatial convergence is demonstrate with the
    solution to an analytic problem. Additionally, $\lambda$ is not directly
    related to the norm of the residual $\| \residual (\vx^m) \|$ so no
    conclusion can be made regarding the behavior of $\lambda$ during the
    \gls{jfnk} iterations \cite{caslJFNK}.

  \subsection{Critical Boron Concentration Search}
    For the simulation of \glspl{pwr}, one is typically not interested in the
    value of $\lambda$. Instead, the critical boron concentration, $B$, is
    desired such that $\lambda=1.0$. While this may be a simple feature, the
    authors have not demonstrated the ability of this new method to solve for
    the critical boron concentration. In similar studies, the calculation of a
    critical boron concentration has been performed using a \gls{jfnk}-like
    method but it required special tuning and consideration \cite{caslJFNK}. The
    typical method for calculating the critical boron concentration is related
    to the Power Iteration method. As the \gls{jfnk} method presented by
    \citeauthor{qe2paper} will have different convergence behavior and require a
    different number of outer iterations the standard calculation technique may
    not be applicable.

  \subsection{\texorpdfstring{$\dtilde$ Formulation}{D~ Formulation}}
    \label{sec:dtilde_formulation}
    A potentially significant problem of the \gls{jfnk} solution method 
    presented by \citeauthor{qe2paper} is the solution method of the \gls{nem}
    coefficients. Typically, the \gls{nem} coefficients, $a_{g,u,n}$, are not
    solved directly and, instead, a $\dtilde$ term is calculated. Consider two
    neighboring nodes. Then, using the $\gls{nem}$ coefficients, the current
    across the dividing surface, $\current_g(u)$, can be calculated and a
    $\dtilde$ can be calculated as
    \begin{equation}
      \label{eq:dtilde}
      \current_g(u) = 
        -2 \left( \frac{h_{\ell+1}}{\overline{D}_{g,\ell+1}} + 
          \frac{h_{\ell}}{\overline{D}_{g,\ell}} \right)^{-1}
          \left( \overline{\phi}_{g,u,\ell+1} -
          \overline{\phi}_{g,u,\ell} \right) + 
        \dtilde_{g,u} \left( \overline{\phi}_{g,u,\ell+1} +
          \overline{\phi}_{g,u,\ell} \right)
    \end{equation}
    where $\current_g(u)$ is the current on the edge between cells $\ell$ and
    $\ell+1$, $\overline{\phi}_{g,u,\ell}$ is the node average scalar flux in
    node $\ell$, and $\dtilde$ is dimensionless. $\dtilde$ is now unique to a
    particular node surface and must be solved during every outer iteration as a
    non-linear calculation. The implementation of $\dtilde$ as in
    \eref{eq:dtilde} has been demonstrated previously \cite{palmtagThesis}.

    Calculation of $\dtilde$ has several benefits compared to directly solving
    the \gls{nem} coefficients. Originally, the form of \eref{eq:dtilde} and the
    nonlinear iteration method was proposed by \citeauthor{smith_nonlinear} to
    save on storage. In the nonlinear iteration method, the \gls{nem}
    coefficients need not be stored.

    When using the $\dtilde$ formulation, the linear system of equations solved
    is simply the finite difference equations for the multigroup neutron
    diffusion equation and $\dtilde$ acts as an addition term to each matrix
    element. This has profound implications for the efficient solution of the
    multigroup neutron diffusion equation.


    % matrix sparsity
    % requires a relatively expensive BICGSTAB or GMRES solve

  \subsection{Comparison to Production-Quality Computer Programs}
    % Ragged core
    % Wielandt Shift
    % Language optimization

