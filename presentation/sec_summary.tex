\section{Summary}
\label{sec:summary}

\begin{frame}{Multigroup Neutron Diffusion Equation}
    \begin{equation}
      \label{eq:multigroup_diffusion}
      \grad \cdot \current_g(\vr) + \Sigma_{r,g}(\vr) \phi_g(\vr)= 
        \frac{\chi_g(\vr)}{\lambda} 
        \sum_{g'=1}^{G} \nu\Sigma_{f,g'}(\vr) \phi_{g'}(\vr) + 
        \sum_{\substack{g'=1 \\ g' \ne g}}^{G} 
        \Sigma_{s,g' \rightarrow g}(\vr) \phi_{g'}(\vr)
    \end{equation}
    \vspace{-2\baselineskip}
    \begin{conditions} % custom environment designed for this purpose
      \vr & spatial position vector, \\
      \current_g(\vr) & net neutron current for energy group $g$ 
        \units{$\frac{1}{\text{cm}^2 \; \text{s}}$}, \\
      \phi_g(\vr) & 
        \parbox[t]{\columnwidth}{fundamental eigenvector,  \\
        scalar neutron flux for energy group $g$
        \units{$\frac{1}{\text{cm}^2 \; \text{s}}$},} \\
      \Sigma_{r,g}(\vr) & macroscopic removal cross section for energy group $g$ 
        \units{$\frac{1}{\text{cm}}$}, \\
      \chi_g(\vr) & fission spectrum for energy group $g$,\\
      \lambda & 
        \parbox[t]{\columnwidth}{fundamental eigenvalue, \\
        effective neutron multiplication factor,} \\
      \nu \Sigma_{f,g}(\vr) & 
        \parbox[t]{\columnwidth}{number of fission neutrons times macroscopic
        fission \\
        cross section in energy group $g$ \units{$\frac{1}{\text{cm}}$},} \\
      \Sigma_{s,g' \rightarrow g} (\vr) & 
        \parbox[t]{\columnwidth}{macroscopic scatter cross section from
        energy group $g'$ to \\
        energy group $g$ \units{$\frac{1}{\text{cm}}$},} \\
      G & total number of energy groups.
    \end{conditions}
\end{frame}

\begin{frame}{\glsentryshort{nem} Equations}
  Transverse integrated multigroup neutron diffusion equation. \\
  Note: node indices $i,j,k$ have been omitted.
  \begin{align}
    \label{eq:transverse_multigroup_diffusion}
    \frac{d \current_{g,u} (u)}{d u} + \overline{\Sigma_{r,g}}
      \phi_{g,u}(u) &= Q_{g,u}(u) - L_{g,u}(u) \\
    \label{eq:current_approximation}
    \current_{g,u}(u) &= - \overline{D_g} \, \frac{d \phi_{g,u}(u)}{du}
  \end{align}
  \vspace{-\baselineskip}
  \begin{conditions}
    u          & coordinate direction (i.e. $u = x,y,z$), \\
    \overline{\Sigma_{r,g}} & average value of $\Sigma_{r,g}(\vr)$ in node $i,j,k$, \\
    \overline{D_g} & average value of diffusion coefficient in node $i,j,k$, \\
    Q_{g,u}(u) & transverse integrated neutron source, \\
    L_{g,u}(u) & transverse leakage.
  \end{conditions}
\end{frame}

\begin{frame}{\glsentryshort{nem} Projections}
  Basis functions are typically polynomials.\\
  \citeauthor{qe2paper} select Legendre polynomials.
  \begin{align}
    \label{eq:flux_expansion}
    \phi_{g,u}(u) &= \sum_{n=0}^{N_{\phi} = 4} a_{g,u,n} \, f_{u,n}(u), \\
    \label{eq:source_expansion}
    Q_{g,u}(u)    &= \sum_{n=0}^{N_Q = 2}      q_{g,u,n} \, f_{u,n}(u), \\
    \label{eq:leakage_expansion}
    L_{g,u}(u)    &= \sum_{n=0}^{N_L = 2}      l_{g,u,n} \, f_{u,n}(u),
  \end{align}
  \vspace{-\baselineskip}
  \begin{conditions}
    a_{g,u,n} & expansion coefficient of $\phi_{g,u}(u)$, \\
    q_{g,u,n} & expansion coefficient of $Q_{g,u}(u)$, \\
    l_{g,u,n} & expansion coefficient of $L_{g,u}(u)$, \\
    f_{u,n}(u) & $n^{th}$ Legendre polynomial.
  \end{conditions}
\end{frame}

\begin{frame}{Local Elimination}
  \begin{itemize}
    \item Odd and even coefficients can be solved separately \cite{gehinThesis}.
    \item \citeauthor{qe2paper} show $a_{g,u,1}$ and $a_{g,u,3}$ can be written
      in terms of each other.
    \item $a_{g,u,1-3}$ and $a_{g,u,2-4}$ are introduced.
    \item Solution vector, $\Phi_g$ is length $10 \times N \times G$.
  \end{itemize}
  \begin{equation}
    \label{eq:solution_vector}
    \vPhi_g =
    \begin{pmatrix}
      \current_{g,x,+} \\
      \current_{g,y,+} \\
      \vspace{8pt}
      \current_{g,z,+} \\
      \overline{\phi_g} \\
      a_{g,x,1-3} \\
      a_{g,y,1-3} \\
      a_{g,z,1-3} \\
      a_{g,x,2-4} \\
      a_{g,y,2-4} \\
      a_{g,z,2-4}
    \end{pmatrix}
  \end{equation}
\end{frame}

\begin{frame}{\glsentryshort{jfnk} Theory}
  The $m^{th}$ Newton step.
  \begin{equation}
    \label{eq:newton_step}
    \jacobian (\vx^m) \cdot \step^m = - \residual(\vx^m)
  \end{equation}
  The step proceeds.
  \begin{equation}
    \vx^{m+1} = \vx^{m} + \step^{m}
  \end{equation}
\end{frame}

\begin{frame}{Inexact Newton Condition}
  The Newton step is solved with a Krylov solver. \\
  \glsentryshort{gmres} and \glsentryshort{bicgstab} are both investigated.
  \begin{equation}
    \label{eq:inexact_newton_condition}
    \| \residual(\vx^m) + \jacobian(\vx^m) \cdot \step^m \| \le 
      \eta_m \| \residual(\vx^m) \|
  \end{equation}
  The Krylov solver does not require an explicit Jacobian, only the
  Jacobian-vector product which can be approximated with finite differences.
  \begin{equation}
    \label{eq:dirder}
    \jacobian(\vx^m) \cdot \vv \approx \frac{\residual(\vx^m + \epsilon \vv) - 
      \residual(\vx^m)}{\epsilon}
  \end{equation}
  Typically, $\epsilon = \sqrt{\epsilon_{mach}} \approx 10^{-8}$.
\end{frame}

\begin{frame}{Choice of Physics-Based Preconditioner}
  \begin{equation}
    \label{eq:left_precondition}
    \| \mm_L^{-1} \residual(\vx^m) + \mm_L^{-1} \left( \jacobian(\vx^m)
      \cdot \step^m \right) \| \le \eta_m \| \mm_L^{-1} \residual(\vx^m) \|
  \end{equation}
  \begin{itemize}
    \item Preconditioner should approximate the Jacobian inverse
      \cite{textbookkelley}.
    \item \citeauthor{gill_azmy} investigate several choices of preconditioner
      and conclude that preconditioning with $\approx 5$~\glspl{pi} is ideal.
    \item \citeauthor{jfnk_wielandt} present similar results.
    \item \citeauthor{qe2paper} develop a preconditioner based on available
      data.
    \item Solved using \gls{tdma} and then \gls{adi} method.
    \item No preconditioner comparison provided.
  \end{itemize}
\end{frame}

\begin{frame}{Convergence Rates of \glsentryshort{jfnk} and \glsentryshort{pi}
  Methods}
  \begin{itemize}
    \item \gls{pi}.
    \begin{itemize}
      \item Converges linearly at a rate determined by the dominance ratio
        \cite{nakamura}.
      \begin{equation}
        d = \frac{\lambda_1}{\lambda_0}
      \end{equation}
      \item Typically, $d > 0.95$ is common and the \gls{ws} is used (to be 
        discussed) \cite{gehinThesis}.
      \begin{equation}
        d' = \frac{\frac{1}{\lambda_0} - \frac{1}{\lambda'}}
          {\frac{1}{\lambda_1} - \frac{1}{\lambda'}}
      \end{equation}
      % if \lambda_0 = 1.0, \lambda_1 = 0.95, \lambda' = \lambda_0 + 0.03
      % d = 0.95, d' = 0.35625
    \end{itemize}
    \item \gls{jfnk}.
    \begin{itemize}
      \item Convergence rate determined by Jacobian properties (e.g. Lipschitz
        constant) \cite{textbookkelley}.
      \item Not affected by dominance ratio \cite{gill_azmy}.
      \item Will not be affected by \gls{ws} despite claim of
        \citeauthor{qe2paper}.
    \end{itemize}
  \end{itemize}
\end{frame}
