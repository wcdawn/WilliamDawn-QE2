\pdfbookmark[section]{Abstract}{toc}
\begin{abstract}

  \gls{jfnk} methods promise a robust framework for solving nonlinear equations.
  The \gls{jfnk} methods provides $q$-quadratic convergence and only require a
  finite difference Jacobian-vector product rather than a full Jacobian.
  However, the adoption of the \gls{jfnk} method in the simulation of nuclear
  reactors has proven difficult due to computational complexity. As such, the
  traditional \gls{pi} method is typically preferred to solve for the
  fundamental eigenvalue and eigenvector of a reactor system.

  In \citetitle{qe2paper}, authors \citeauthor{qe2paper} present an
  implementation of the \gls{jfnk} method to solve the \gls{nem} equations. In
  their method, the authors reduce the number of solution variables by employing
  local-elimination and improve the efficiency of the Krylov solver by
  preconditioning with a specially designed physics-based preconditioner. This
  implementation of the \gls{jfnk} method is compared to a \gls{pi} method and
  the authors conclude that ``the \gls{nem}\_\gls{jfnk} methods can greatly
  improve the convergence rate and the computational efficiency''
  \cite{qe2paper}.
  
  The claims by the authors are exaggerated and insufficient data is provided to
  justify such claims. The claim of improved computational efficiency is based
  on comparing the proposed \gls{jfnk} method to a suboptimal implementation of
  the \gls{pi} method. Notably, the authors do not properly consider the
  importance of the \gls{ws} and do not use a $\dtilde$ formulation of the
  \gls{nem} equations. In their closing remarks, \citeauthor{qe2paper} also
  claim that the proposed methods can be extended to multiphysics simulations
  though other work suggests this is challenging, if at all feasible
  \cite{caslJFNK}.

  Ultimately, \citeauthor{qe2paper} present a comparison of suboptimal codes and
  suggest that the \gls{jfnk} method is preferable because it has reduced
  execution time. Such claims cannot be substantiated by the results presented.
  It is essential to compare optimized codes before making conclusions regarding
  an improvement in methods.

\end{abstract}
\glsresetall
