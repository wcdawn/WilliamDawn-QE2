\section{Summary and Background}
\label{sec:summary}

  A brief summary of \citetitle{qe2paper} follows. Supporting information and 
  commentary on the methods presented is included as applicable.

  \subsection{Multigroup Neutron Diffusion and \glsentrylong{nem}}

    % write the equation
    In its conventional form, the multigroup neutron diffusion equation for a
    problem domain $\vr \in \Omega$ can be written 
    \begin{equation}
      \label{eq:multigroup_diffusion}
      \grad \cdot \current_g(\vr) + \Sigma_{t,g}(\vr) \phi_g(\vr)= 
        \frac{\chi_g(\vr)}{\lambda} 
        \sum_{g'=1}^{G} \nu\Sigma_{f,g'}(\vr) 
        \phi_{g'}(\vr) + \sum_{g'=1}^{G} \Sigma_{s,g' \rightarrow g}(\vr) 
        \phi_{g'}(\vr)
    \end{equation}
    where 
    \begin{conditions} % custom environment designed for this purpose
      \vr & spatial position vector, \\
      \current_g(\vr) & net neutron current for energy group $g$ 
        \units{$\frac{1}{\text{cm}^2 \; \text{s}}$}, \\
      \phi_g(\vr) & fundamental eigenvector, 
        scalar neutron flux for energy group $g$
        \units{$\frac{1}{\text{cm}^2 \; \text{s}}$}, \\
      \Sigma_{t,g}(\vr) & macroscopic total cross section for energy group $g$ 
        \units{$\frac{1}{\text{cm}}$}, \\
      \chi_g(\vr) & fission spectrum for energy group $g$,\\
      \lambda & fundamental eigenvalue, effective neutron multiplication factor, \\
      \nu \Sigma_{f,g}(\vr) & number of fission neutrons times macroscopic fission
        cross section in energy group $g$ \units{$\frac{1}{\text{cm}}$}, \\
      \Sigma_{s,g' \rightarrow g} (\vr) & macroscopic scatter cross section from
        energy group $g'$ to energy group $g$ \units{$\frac{1}{\text{cm}}$}, \\
      G & total number of energy groups.
    \end{conditions}

    The three-dimensional domain $\Omega$ can then be discretized into a
    structured grid of nodes. Next, the \gls{nem} is introduced. Fundamentally,
    the \gls{nem} projects the transverse integrated flux onto a set of basis
    functions. The transverse integrated multigroup neutron equation is given as
    \cite{qe2paper}.
    \begin{equation}
      \label{eq:transverse_multigroup_diffusion}
      \frac{d \, \current_{g,u} (u)}{d u} + \overline{\Sigma_{r,g}} \,
        \phi_{g,u}(r) = Q_{g,u}(u) - L_{g,u}(u)
    \end{equation}
    where $u = x,\ y,\ z$ is a coordinate direction. The terms $Q_{g,u}(u)$ and
    $L_{g,u}$ represent the transverse integrated neutron source and transverse
    leakages respectively. Note, in \eref{eq:transverse_multigroup_diffusion}
    that the node indices (i.e. $i,j,k$) have been omitted and
    $\overline{\Sigma_{r,g}}$ represents the average value of
    $\Sigma_{r,g}(\vr)$ in the node. The transverse integrated current
    $\current_{g,u}$ is then approximated
    \begin{equation}
      \label{eq:current_approximation}
      \current_{g,u} = - \overline{D_g} \, \frac{d \phi_{g,u}(u)}{du}
    \end{equation}
    where $\overline{D_g}$ is the average value of the diffusion coefficient in
    the node.

    \gls{nem} basis functions are typically polynomials and
    \citeauthor{qe2paper} select the Legendre polynomials. In their work, the
    scalar flux is projected onto quartic polynomials and the transverse
    integrated neutron source and transverse leakage terms are projected onto
    quadratic polynomials. The selection of quartic and quadratic polynomials
    respectively is consistent as the second derivative of the scalar flux is
    related to the source and leakage terms \cite{gehinThesis}. This projection
    can be written
    \begin{align}
      \label{eq:flux_expansion}
      \phi_{g,u}(u) &= \sum_{n=0}^{N_{\phi} = 4} a_{g,u,n} \, f_{u,n}(u), \\
      \label{eq:source_expansion}
      Q_{g,u}(u)    &= \sum_{n=0}^{N_Q = 2}      q_{g,u,n} \, f_{u,n}(u), \\
      \label{eq:leakage_expansion}
      L_{g,u}(u)    &= \sum_{n=0}^{N_L = 2}      l_{g,u,n} \, f_{u,n}(u),
    \end{align}
    where $a_{g,u,n}$, $q_{g,u,n}$, and $l_{g,u,n}$ are the expansion
    coefficients of $\phi_{g,u}(u)$, $Q_{g,u}(u)$, and $L_{g,u}(u)$ respectively
    and $f_{u,n}(u)$ is the $n^{th}$ Legendre polynomial \cite{qe2paper}. To
    solve for the coefficients of the expansions in \eref{eq:flux_expansion},
    \eref{eq:source_expansion}, and \eref{eq:leakage_expansion}, five equations
    are required. The selected equations are:
    \begin{enumerate}
      \item flux continuity condition,
      \item current continuity condition,
      \item nodal neutron balance (also $f_0$ weighted residual).
      \item $f_1$ weighted residual, and
      \item $f_2$ weighted residual.
    \end{enumerate}
    Special attention is paid to boundary conditions and these conditions 
    replace the flux and current continuity equations.

    \subsubsection{Local Elimination}
      \label{sec:local_elimination}

      Recall the quantities of interest for this calculation are the fundamental
      eigenmode composed of eigenvalue $\lambda$ and eigenvector $\phi_g(\vr)$.
      The coefficients in \eref{eq:flux_expansion}, \eref{eq:source_expansion},
      and \eref{eq:leakage_expansion} are not desired directly, but must be
      solved in order to obtain the quantities of interest. It has been
      previously demonstrated that all coefficients need not be solved
      simultaneously which allows for the reduction of the dimensionality of the
      problem \cite{gehinThesis}.

      The zeroth coefficient $a_{g,u,0} = \overline{\phi_{g,u}}$ because $f_0 =
      1$. Therefore this, variable cannot be eliminated as it is one of the
      quantities of interest. Then, the odd and even coefficients can be solved
      separately \cite{gehinThesis}. This is a result of the choice of the five
      solution equations selected. 

      \citeauthor{qe2paper} then show that the number of coefficients can be
      further reduced as the odd coefficients can be expressed in terms of
      each other with similar results for the even coefficients. That is,
      instead of solving for $a_{g,u,1}$ and $a_{g,u,3}$, a new coefficient is
      introduced $a_{g,u,1-3}$. Similarly, $a_{g,u,2-4}$ becomes a solution
      variable. This is an important and novel step of the reviewed work as the
      solution of \gls{nem} coefficients represents the bulk of the
      computational time and the number of coefficients for each node and each
      energy group has been reduced from five to three
      \cite{qe2paper}.

      Finally, the \gls{nem} equations are solved to obtain currents, scalar
      fluxes, $a_{g,u,1-3}$, and $a_{g,u,2-4}$ in each node for each energy
      group. The resulting solution vector, $\vPhi_g$, can be written 
      \begin{equation}
        \label{eq:solution_vector}
        \vPhi_g =
        \begin{pmatrix}
          \current_{g,x,+} \\
          \current_{g,y,+} \\
          \vspace{8pt}
          \current_{g,z,+} \\
          \overline{\phi_g} \\
          a_{g,x,1-2} \\
          a_{g,y,1-2} \\
          a_{g,z,1-2} \\
          a_{g,x,2-4} \\
          a_{g,y,2-4} \\
          a_{g,z,2-4}
        \end{pmatrix}
      \end{equation}
      and the total number of solving variables is $10 \times N \times G$ where
      $N$ is the total number of nodes and $G$ is the total number of energy
      groups. Note that the \gls{nem} coefficients and neutron current must be
      solved in each spatial direction. Typically, solving for the \gls{nem}
      coefficients themselves is not preferred as the values themselves are not
      quantities of interest.  The preferred implementation computes a modified
      diffusion coefficient, $\dtilde$, and will be discussed in
      \sref{sec:dtilde_formulation}.

    \subsubsection{Choice of Physics-Based Preconditioner}

      Ultimately, the \gls{nem} coefficients will be solved iteratively with a
      Krylov solver. The wise choice of preconditioner can significantly reduce 
      the number of Krylov iterations to solve a linear system and it can be
      shown that if one preconditions a matrix with its inverse, the Krylov
      method will converge in one iteration \cite{textbookkelley}. While
      computing the matrix inverse would defeat the purpose of the Krylov
      method, this result indicates that the preconditioning operator is
      selected to approximate the matrix inverse.

      An extensive review of the choice of preconditioner in \gls{jfnk}
      solutions to the multigroup neutron diffusion equation has been presented
      elsewhere \cite{gill_azmy}. It is expected that the results of the
      preconditioner study for the neutron diffusion equation should be
      applicable to the \gls{nem} equations. Preconditioners investigated
      include the diffusion operator, incomplete Cholesky factorization, the
      diffusion-fission operator, and preconditioning with a few \glspl{pi}.
      The results of the preconditioner study indicate that preconditioning with
      approximately five \glspl{pi} provided the fastest convergence of the
      Newton iterations. The incomplete Cholesky preconditioner provided similar
      convergence rate but requires additional and expensive matrix
      calculations. A \gls{pi} preconditioner was also investigated in work by
      \citeauthor{jfnk_wielandt} and similar performance was reported.

      The use of a \gls{pi} preconditioner was not investigated by
      \citeauthor{qe2paper}. Instead, the authors design a new
      preconditioner based on the data available during the \gls{nem}
      iterations. The benefit of this is that no additional data is required.
      Special attention is also paid to ensure the preconditioner can be solved
      using the \gls{tdma} and then \gls{adi} methods. These methods are among
      the fastest numerical matrix solution techniques available. The authors
      incorporate information in the preconditioner from the \gls{nem}
      equations, group-to-group scattering, and transverse leakage
      contributions. The construction of the preconditioner appears to be 
      computationally expensive. Unfortunately, \citeauthor{qe2paper} do not
      compare their preconditioner to any other preconditioner or even a case
      without a preconditioner. Therefore, it is impossible to determine the
      relative computational efficiency of this preconditioning method.

  \subsection{\glsentrylong{jfnk} Theory and Inexact Newton Condition}

    The \gls{jfnk} method begins with Newton's method. Newton's method provides
    a framework to find the root of a residual function 
    $\residual(\vx^*) = \vzero$ where $\vx^*$ is a root of the residual 
    function. It is first necessary to construct a residual function based on
    the \gls{nem} equations. \citeauthor{qe2paper} propose a residual function
    that contains the \gls{nem} solution, $\vPhi$, residual and the absolute
    change in $\lambda$ during the Newton iteration. This differs from the
    residual function proposed by \citeauthor{gill_azmy} but the new form may be
    preferable as it directly contains information about $\lambda$.
    
    In vector space, the Newton step at the $m^{th}$ step can be described 
    \begin{equation}
      \label{eq:newton_step}
      \jacobian (\vx^m) \, \step^m = - \residual(\vx^m)
    \end{equation}
    where $\jacobian(\vx^m)$ is the Jacobian of $\residual(\vx^m)$ and 
    $\step^m$ is the step. Then, the Newton step proceeds as
    \begin{equation}
      \vx^{m+1} = \vx^{m} + \step^{m}.
    \end{equation}

    For the choice of residual function made by \citeauthor{qe2paper}, an 
    analytic Jacobian is not simple to compute. Therefore, a finite difference
    approximation is used to estimate the Jacobian. In fact, the Jacobian itself
    is not needed as the Newton step in \eref{eq:newton_step} will be solved
    with a Krylov method. All that is needed is the finite difference
    Jacobian-vector product which is equivalent to a directional derivative. The
    directional derivative can be written
    \begin{equation}
      \label{eq:dirder}
      \jacobian(\vx^m) \, \vv \approx \frac{\residual(\vx^m + \epsilon \vv) - 
        \residual(\vx^m)}{\epsilon}
    \end{equation}
    where $\vv$ is the direction (typically $\step^m$) and $\epsilon$ is the
    finite difference step size. Typically, $\epsilon$ is on the order of
    machine precision, $\epsilon_{mach} \approx 10^{-16}$
    \cite{qe2paper,gill_azmy,textbookkelley}. Since the Jacobian has been
    replaced by a directional derivative, the method is termed Jacobian-Free.

    As previously mentioned, the Newton step in \eref{eq:newton_step} is solved
    using a Krylov method as
    \begin{equation}
      \label{eq:inexact_newton_condition}
      \| \residual(\vx^m) + \jacobian(\vx^m) \, \step^m \| \le 
        \eta_m \| \residual(\vx^m) \|
    \end{equation}
    where $\eta_m$ is the forcing term. \citeauthor{qe2paper} investigate
    solving \eref{eq:inexact_newton_condition} using \gls{gmres} and
    \gls{bicgstab} Krylov methods. \eref{eq:inexact_newton_condition} is termed
    the inexact Newton condition. It is known that the choice of forcing term,
    $\eta_m$, can significantly effect the convergence rate of the \gls{jfnk}
    \cite{textbookkelley}. Essentially, it is undesirable to ``over-solve''
    \eref{eq:inexact_newton_condition}, especially during the initial Newton
    iterations.

    The Eisenstat-Walker forcing term is a standard choice for $\eta_m$ and has
    been shown to work well \cite{qe2paper,gill_azmy}. In both the work by
    \citeauthor{qe2paper} and the work by \citeauthor{gill_azmy}, the authors
    perform a study to determine an preferable choice of forcing term. The
    authors all determine that the Eisenstat-Walker forcing term is ideal.
    However, $\eta_m = 10^{-1}$ provides equally good, and sometimes better
    results \cite{qe2paper,gill_azmy,jfnk_wielandt,ma784notes}. These results 
    indicate that both the Eisenstat-Walker forcing term and $\eta_m = 10^{-1}$
    perform similarly and such a study is unnecessary in the future.

  \subsection{Convergence Rate Consideration}

    The $q$-quadratic convergence rate of the \gls{jfnk} method is what makes it
    desirable to implement. Therefore, it is important to compare the
    convergence behavior of the \gls{jfnk} method to the currently common
    \gls{pi} method.

    \subsubsection{\texorpdfstring{Convergence of \glsentrylong{pi}
      Method}{Convergence of Power Iteration Method}}
      \label{sec:dominance_ratio}

      The convergence behavior of the \gls{pi} method is known and well studied 
      \cite{nakamura,gehinThesis,my_ms_thesis}. The convergence rate of the
      \gls{pi} method is determined by the first and second largest eigenvalues
      of the equations of interest (either the multigroup neutron diffusion
      equation or \gls{nem} equations). It can be shown that each iteration of
      the \gls{pi} method reduces the error by a multiple termed the dominance
      ratio, $d$, defined as
      \begin{equation}
        \label{eq:dominance_ratio}
        d = \frac{\lambda_1}{\lambda_0}
      \end{equation}
      where $\lambda_0 = \lambda$ is the fundamental eigenvalue and $\lambda_1$
      is the next largest eigenvalue \cite{my_ms_thesis}. The \gls{pi} method
      converges $q$-linearly and the dominance ratio is the leading coefficient
      in the convergence behavior. $d$ is guaranteed to be less than one and
      smaller values of $d$ result in faster convergence \cite{nakamura}. For
      typical reactor simulations, $d > 0.9$ and often $d > 0.95$ implying
      convergence is slow, often requiring hundreds of iterations.

      The typical approach to address the challenge of large dominance ratios is
      the \gls{ws} \cite{gehinThesis}. Briefly, the \gls{ws} subtracts a portion
      of the fission source from both sides of the equation. It can be shown
      that the dominance ratio is decreased \textit{significantly} such that $d
      \approx 0.5$ may be expected. It is common for the \gls{ws} to speedup 
      the solution \gls{pi} method by a factor of three or more. This will be
      discussed more in \sref{sec:wielandt_shift}.

    \subsubsection{Convergence of Newton Method}

      When the iterate in the \gls{jfnk} method, $\vx^m$, is far from the
      solution, $\vx^*$, the method converges $q$-superlinearly. As the
      iteration approaches the solution, it achieves $q$-quadratic convergence
      \cite{textbookkelley}.

      It has been demonstrated that the dominance ratio of the multigroup
      neutron diffusion equation has little effect on the convergence rate when
      the equations are solved using the \gls{jfnk} method \cite{gill_azmy}.
      This is expected as the convergence of the \gls{jfnk} method is not
      related to the eigenvalues of the operator, but instead is related to the
      properties of the Jacobian including its Lipschitz continuity
      \cite{textbookkelley}. These results indicate that the \gls{ws} would have
      little to no effect on the \gls{jfnk} method. However,
      \citeauthor{qe2paper} indicate in their concluding remarks that they
      expect the \gls{ws} to ``improve the efficiency further'' and this is
      inaccurate \cite{qe2paper}.
      
