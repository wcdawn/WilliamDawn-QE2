\section{Introduction}
\label{sec:introduction}

  The multigroup neutron diffusion equation has proven to be a useful tool for
  modeling for the neutron distribution in a typical nuclear reactor. With the
  development of the \gls{nem}, the model was made more accurate and
  computational time was significantly decreased. The typical quantities of
  interest in these reactor simulations is the fundamental mode of the
  eigenvalue problem; that is, the largest eigenvalue and associated scalar flux
  distribution.

  Traditionally, the multigroup neutron diffusion equation and \gls{nem}
  equations have been solved using the \gls{pi} method. The \gls{pi} method is a
  fixed-point method useful for solving eigenvalue problems. It is a linear
  iteration method and converges to the fundamental eigenmode $q$-linearly.
  Recently, the use of \gls{jfnk} methods has been investigated in an attempt to
  reduce the computing time required to solve the \gls{nem} equations.
  \gls{jfnk} methods are attractive because when the iteration terminates, the
  method converges $q$-quadratically. However, \gls{jfnk} methods require the
  computation of quantities not required by the \gls{pi} method including the
  finite difference approximate Jacobian-vector product. Due to the challenges
  of computing these quantities, the implementation of \gls{jfnk} methods for
  computing the fundamental mode to the multigroup neutron diffusion equation
  remains an active area of research.

  In \citetitle{qe2paper}, \citeauthor{qe2paper} explore an implementation of
  the \gls{jfnk} method that seeks to efficiently solve the \gls{nem} equations
  for the fundamental eigenmode. \sref{sec:summary} provides a brief summary of
  this work and a few considerations for solving the \gls{nem} equations using
  the \gls{jfnk} method are provided. Then, \sref{sec:critique} critiques the
  authors' implementation with a focus on the applicability of the proposed
  method to realistic nuclear reactor simulations. Finally,
  \sref{sec:conclusion} provides a few conclusions based on the work by
  \citeauthor{qe2paper} and some considerations for future implementation of the 
  \gls{jfnk} method to solve the \gls{nem} equations.
