\section{Introduction}
\label{sec:introduction}

% motivation for JFNK investigation
The multigroup neutron diffusion equation has been demonstrated as a useful
model for the neutron distribution in a typical nuclear power reactor. With the
development of the \gls{nem}, the model was made more efficient and
computational time was significantly decreased. The typical quantities of
interest in these reactor simulations is the fundamental mode of the eigenvalue
problem; that is, the largest eigenvalue and associated scalar flux
distribution.

Traditionally, the multigroup neutron diffusion equation has been solved using
the Power Iteration method. The Power Iteration method is a fixed-point method
useful for solving eigenvalue problems. It is a linear iteration method and
converges to the fundamental eigenmode with $q$-linear convergence rate.
Recently, work has investigated the use of \gls{jfnk} methods in an attempt to
improve the computational efficiency of solving the multigroup neutron diffusion
equation. \gls{jfnk} methods are attractive because when the iteration
terminates, the method has $q$-quadratic convergence rate. However, \gls{jfnk}
methods require the computation of quantities not required by the Power
Iteration method including the finite difference approximate Jacobian. Due to
the challenges of computing these quantities, the implementation of \gls{jfnk}
methods for computing the fundamental mode to the multigroup neutron diffusion
equation remains an active area of research.

% outline of sections
In \citetitle{qe2paper}, \citeauthor{qe2paper} explore an implementation of the
\gls{jfnk} that seeks to be efficiently solve the \gls{nem} equations for the
fundamental eigenmode. \sref{sec:summary} provides a brief summary of this
work and a few considerations for solving the multigroup neutron diffusion
equation using \gls{jfnk} methods. Then, \sref{sec:critique} critiques the
author's implementation with a focus on the applicability of the proposed method
to realistic nuclear power reactor simulations. Finally, \sref{sec:conclusion}
provides a few conclusions based on the work by \citeauthor{qe2paper} and
provides an some considerations for future implementation of the \gls{jfnk}
method.
