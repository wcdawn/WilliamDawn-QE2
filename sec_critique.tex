\section{Critique}
\label{sec:critique}

  The primary motivation for this work is to critique the article by
  \citeauthor{qe2paper}. While there may be more to critique, the discussion
  here focuses on the application of proposed the \gls{jfnk} method to realistic
  reactor simulation problems.

  \subsection{Verification and Validation}

    Had the authors provided additional results, the implications of the work
    may be more significant. To demonstrate the implementation of their method,
    the authors compare their converged eigenvalue and scalar flux distribution
    to a single benchmark problem: the \gls{iaea} 3D \gls{pwr} benchmark. While
    results are presented for a second pebble bed reactor, these results have
    not been previously verified. The authors do not demonstrate the spatial
    convergence order of their method. Typically spatial convergence is
    demonstrated with the solution to an analytic problem and a mesh convergence
    study. Additionally, the fundamental eigenvalue, $\lambda$, is not directly
    related to the norm of the residual, $\| \residual (\vx^m) \|$, so no
    conclusion can be made regarding the behavior of $\lambda$ during the
    \gls{jfnk} iterations~\cite{caslJFNK}.

  \subsection{Critical Boron Concentration Search}

    For the simulation of \glspl{pwr}, one is typically not interested in the
    value of $\lambda$. Instead, the critical boron concentration is desired
    such that $\lambda=1.0$. While this may be a simple feature, the authors
    have not demonstrated the ability of this new method to solve for the
    critical boron concentration. In similar studies, the calculation of a
    critical boron concentration has been performed using a \gls{jfnk}-like
    method but it required special tuning and consideration \cite{caslJFNK}. The
    typical method for calculating the critical boron concentration is related
    to the \gls{pi} method. The \gls{jfnk} method presented by
    \citeauthor{qe2paper} will have different convergence behavior compared to
    the conventional \gls{pi} method so the standard calculation technique may
    not be applicable.

  \subsection{\glsentrylong{cmfd} Formulation}
    \label{sec:cmfd_formulation}

    A significant challenge of the \gls{jfnk} method presented by
    \citeauthor{qe2paper} is the solution of the \gls{nem} coefficients.
    Typically, the \gls{nem} coefficients, $a_{g,u,n}$, are not solved directly
    and, instead, a modified diffusion coefficient, $\dtilde$, term is
    calculated. Consider two neighboring nodes, indexed $\ell$ and $\ell+1$.
    Then, using the \gls{nem} coefficients calculated in only these two nodes,
    the current across the intervening surface, $\current_{g,u}(u)$, can be
    calculated. A $\dtilde$ can be calculated as
    \begin{equation}
      \label{eq:dtilde}
      \current_{g,u}(u) = 
        -2 \left( \frac{h_{\ell+1}}{\overline{D}_{g,\ell+1}} + 
          \frac{h_{\ell}}{\overline{D}_{g,\ell}} \right)^{-1}
          \left( \overline{\phi}_{g,u,\ell+1} -
          \overline{\phi}_{g,u,\ell} \right) + 
        \dtilde_{g,u} \left( \overline{\phi}_{g,u,\ell+1} +
          \overline{\phi}_{g,u,\ell} \right)
    \end{equation}
    where $\current_{g,u}(u)$ is the current on the surface between the nodes,
    $\overline{\phi}_{g,u,\ell}$ is the node average scalar flux in node $\ell$,
    and $\dtilde$ is dimensionless. $\dtilde$ is now unique to a particular
    surface and must be solved during every outer iteration as a nonlinear
    update. The implementation of $\dtilde$ as in \eref{eq:dtilde} has been
    demonstrated previously \cite{smith_nonlinear,palmtagThesis}. Note that all
    demonstrations of the \gls{cmfd} formulation and proofs relating to the
    nonlinear iteration process are related to the \gls{pi} method. Implementing
    the \gls{cmfd} formulation with the \gls{jfnk} method may present additional
    challenges.

    Calculation of $\dtilde$ has several benefits compared to directly solving
    for the \gls{nem} coefficients. Originally, the form of \eref{eq:dtilde} and
    the nonlinear iteration method was proposed by \citeauthor{smith_nonlinear}
    to save on storage. In the nonlinear iteration method, the \gls{nem}
    coefficients need not be stored. This \gls{cmfd} formulation would reduce
    the number of solution variables from ${10 \times N \times G}$ to 
    ${N \times G}$. The bulk of the computation in each Newton iteration is
    spent in the Krylov solver to solve the inexact Newton condition from
    \eref{eq:inexact_newton_condition}. Reducing the dimension of the linear
    operator by implementing a \gls{cmfd} formulation will therefore reduce the
    computational expense of the matrix-vector products required by the Krylov
    solver.

    A final consideration for the \gls{cmfd} formulation is the incorporation
    into existing reactor simulation computer programs. A majority of computer
    programs developed to simulate nuclear reactors use a $\dtilde$ term whether
    it be calculated using the \gls{nem} or another, higher-order, method such
    as \gls{dsa} \cite{casmo4,simulate3,mpact}. By solving for the \gls{nem}
    coefficients themselves, the method proposed by \citeauthor{qe2paper}
    requires the development of an entirely new computer program.

  \subsection{Comparison to Production-Quality Computer Programs}

    The most significant challenge to analyzing the results presented by
    \citeauthor{qe2paper} is that the results of the \gls{jfnk} method are not
    compared to ``production-quality'' computer programs. 
    
    \subsubsection{Ragged Core}

      It almost goes without saying that solution time can be expected to be
      directly related to the number of solving variables. However,
      \citeauthor{qe2paper} include unnecessary nodes in their simulation. The
      simulations presented are in the domain of a rectangular prism but
      standard reactor simulations are performed on a ``ragged core'' domain. In
      order to simplify the boundary description, the authors introduce new
      nodes at the periphery of the problem. Consider the \gls{iaea} 3D
      \gls{pwr} benchmark. The introduction of new nodes does not significantly
      affect the result as the simulation agrees to the benchmark within
      $4~\units{pcm}$. However, the benchmark geometry can be described exactly
      by 245, $10\units{cm} \! \times \! 10\units{cm}$ nodes at each axial
      elevation and the authors simulate 289 nodes at each axial elevation. The
      authors went to great lengths to reduce the number of solution variables
      (see \sref{sec:local_elimination}), but approximately 18\% of the nodes
      solved are absolutely unnecessary for the problem.

      It may seem that simulating a ragged core would only further improve the
      relative speedup of the \gls{jfnk} method compared to the \gls{pi} method,
      but such a conclusion is not straightforward. It is known that in reactor
      simulations using the \gls{pi} method, the method may spend many
      iterations unnecessarily solving for the flux in cells with little or no
      fissile material \cite{gehinThesis}. In the work by \citeauthor{qe2paper},
      the convergence of the quantities of interest, $\lambda$ and
      $\overline{\phi}$, cannot be inferred. Therefore, it is possible that the
      \gls{pi} method may be spending many iterations solving for the scalar
      flux in these unnecessary nodes whereas the \gls{jfnk} method would not
      experience such behavior.

    \subsubsection{\glsentrylong{ws}}
    \label{sec:wielandt_shift}

      The final shortcoming of the comparison of the \gls{jfnk} and \gls{pi}
      methods presented by \citeauthor{qe2paper} is the improper consideration
      of the \gls{ws}. The authors claim that integrating the \gls{ws} method
      into the \gls{jfnk} method will ``improve efficiency further''
      \cite{qe2paper}. However, both the work by \citeauthor{gill_azmy} and the
      discussion of the \gls{pi} dominance ratio in \sref{sec:dominance_ratio}
      indicate that this claim is misguided. It is not expected that
      implementing the \gls{ws} will improve the efficiency of the \gls{jfnk}
      method.

      The greatest difficulty of the results presented by \citeauthor{qe2paper}
      is that the \gls{pi} method to which the results were compared did not
      implement the \gls{ws}. It is expected that the \gls{ws} can speedup
      results of the \gls{pi} method for these problems by a factor of three or
      more. This would invalidate some of the conclusions by
      \citeauthor{qe2paper} because the \gls{pi} method would then be faster
      than the \gls{jfnk} method for some cases.

      Results by \citeauthor{jfnk_wielandt} indicate that even with the
      \gls{ws}, the \gls{jfnk} method may still provide speedup for the
      multigroup neutron diffusion equation solution to the IAEA benchmark in
      two spatial dimensions compared to the \gls{pi} method
      \cite{jfnk_wielandt}. However, these results do not directly extend to the
      work by \citeauthor{qe2paper} as the solution is presented for the
      \gls{nem} and a heavily modified and specially preconditioned form at
      that.

      Including the \gls{ws} in the \gls{jfnk} method has also been shown to be
      useful for preconditioning the Krylov solver with a series of \glspl{pi}
      \cite{jfnk_wielandt}. Unlike \citeauthor{gill_azmy} and
      \citeauthor{jfnk_wielandt}, \citeauthor{qe2paper} do not investigate
      \gls{pi} preconditioning but such a preconditioner warrants investigation.
