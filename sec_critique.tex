\section{Critique}
\label{sec:critique}
  The primary motivation for this paper is to critique the article by
  \citeauthor{qe2paper}. While there may be more to critique, the discussion
  here focuses on the application of proposed the \gls{jfnk} to realistic
  reactor simulation problems.

  \subsection{Verification and Validation}
    Had the authors provided additional results, the implications of the
    publication may be more significant. However, important data has not been
    reported by the authors. To demonstrate the implementation of their method,
    the authors compare their converged eigenvalue and scalar flux distribution 
    to a single benchmark problem: the \gls{iaea} 3D \gls{pwr} problem. While
    results are presented for a second pebble bed reactor, these results have
    not been previously verified so the authors verify by comparing their
    \gls{jfnk} solution a similar version of the same code. The authors do not
    demonstrate the spatial convergence order of their method. Typically spatial
    convergence is demonstrated with the solution to an analytic problem and a
    mesh convergence study. Additionally, $\lambda$ is not directly related to
    the norm of the residual $\| \residual (\vx^m) \|$ so no conclusion can be
    made regarding the behavior of $\lambda$ during the \gls{jfnk} iterations
    \cite{caslJFNK}.

  \subsection{Critical Boron Concentration Search}
    For the simulation of \glspl{pwr}, one is typically not interested in the
    value of $\lambda$. Instead, the critical boron concentration, $B$, is
    desired such that $\lambda=1.0$. While this may be a simple feature, the
    authors have not demonstrated the ability of this new method to solve for
    the critical boron concentration. In similar studies, the calculation of a
    critical boron concentration has been performed using a \gls{jfnk}-like
    method but it required special tuning and consideration \cite{caslJFNK}. The
    typical method for calculating the critical boron concentration is related
    to the \gls{pi} method. As the \gls{jfnk} method presented by
    \citeauthor{qe2paper} will have different convergence behavior and require a
    different number of outer iterations the standard calculation technique may
    not be applicable.

  \subsection{\texorpdfstring{$\dtilde$ Formulation}{D\textasciitilde \ Formulation}}
    \label{sec:dtilde_formulation}
    A potentially significant problem of the \gls{jfnk} solution method 
    presented by \citeauthor{qe2paper} is the solution method of the \gls{nem}
    coefficients. Typically, the \gls{nem} coefficients, $a_{g,u,n}$, are not
    solved directly and, instead, a $\dtilde$ term is calculated. Consider two
    neighboring nodes, indexed $\ell$ and $\ell+1$. Then, using the $\gls{nem}$
    coefficients, the current across the intervening surface, $\current_g(u)$, 
    can be calculated and a $\dtilde$ can be calculated as
    \begin{equation}
      \label{eq:dtilde}
      \current_g(u) = 
        -2 \left( \frac{h_{\ell+1}}{\overline{D}_{g,\ell+1}} + 
          \frac{h_{\ell}}{\overline{D}_{g,\ell}} \right)^{-1}
          \left( \overline{\phi}_{g,u,\ell+1} -
          \overline{\phi}_{g,u,\ell} \right) + 
        \dtilde_{g,u} \left( \overline{\phi}_{g,u,\ell+1} +
          \overline{\phi}_{g,u,\ell} \right)
    \end{equation}
    where $\current_g(u)$ is the current on the edge between nodes,
    $\overline{\phi}_{g,u,\ell}$ is the node average scalar flux in node $\ell$,
    and $\dtilde$ is dimensionless. $\dtilde$ is now unique to a particular node
    surface and must be solved during every outer iteration as a non-linear
    calculation. The implementation of $\dtilde$ as in \eref{eq:dtilde} has been
    demonstrated previously \cite{smith_nonlinear,palmtagThesis}.

    Calculation of $\dtilde$ has several benefits compared to directly solving
    for the \gls{nem} coefficients. Originally, the form of \eref{eq:dtilde} and
    the nonlinear iteration method was proposed by \citeauthor{smith_nonlinear}
    to save on storage. In the nonlinear iteration method, the \gls{nem}
    coefficients need not be stored.

    When using the $\dtilde$ formulation, the linear system of equations solved
    is a modified form the finite difference equations for the multigroup
    neutron diffusion equation and $\dtilde$ acts as an addition term to each
    matrix element. This has profound implications for the efficient solution of
    the multigroup neutron diffusion equation. The sparsity pattern of the
    matrix in the work by \citeauthor{qe2paper} has been shown to have irregular
    structure \cite{palmtagThesis}. Therefore, the Krylov solver must be
    applicable for general matrices and the authors investigate \gls{bicgstab}
    and \gls{gmres} solvers. However, if the $\dtilde$ formulation is used and
    each energy group is solved individually, a much more efficient \gls{cg}
    solver can be used because the matrix will be \gls{spd}. Additionally,
    solving each group individually is more efficient for many group problems
    \cite{my_ms_thesis}.

    A final consideration for the $\dtilde$ formulation is the incorporation
    into existing multigroup neutron diffusion reactor simulation computer
    programs \cite{casmo4,simulate3,mpact}. A majority of computer programs
    developed to simulate nuclear reactors use a $\dtilde$ term whether it be
    calculated using the \gls{nem} or an other method such as \gls{cmfd}. By
    solving for the \gls{nem} coefficients themselves, the method proposed by
    \citeauthor{qe2paper} requires the development of an entirely new computer
    program. For the \gls{jfnk} method to become widely accepted, it must be
    implemented into existing computer programs to take advantage of existing
    thermal hydraulic, thermal expansion, and other reactor models. Using this
    nonstandard form of the \gls{nem} makes the proposed method difficult to
    compare to the state-of-the-art.

  \subsection{Comparison to Production-Quality Computer Programs}
    The most significant challenge to analyzing the results presented by
    \citeauthor{qe2paper} is that the results of the \gls{jfnk} method are not
    compared to ``production-quality'' computer programs. 
    
    For example, reactor simulators based on the \gls{nem} and using the power
    iteration method, such as SIMULATE-3, have execution times much quicker than
    the time required by the \gls{pi} reported in the article
    \cite{simulate3,qe2paper}. The authors rightfully spent time to optimize the
    \gls{jfnk} method; but, it seems the reference power iteration method to
    which the authors compare is unoptimized. Therefore, the authors compare
    optimized code to unoptimized code and conclude that the optimized code is
    more efficient.
    
    The simulation described in the article being reviewed was written in MATLAB
    which benefits from rapid development but cannot be optimized in the same
    way a compiled language, such as Fortran, can. For example, when a similar
    comparison of the power iteration method and the \gls{jfnk} method for the
    multigroup neutron diffusion equation was performed using a simulation
    written in Fortran, the results indicated the benefits of the \gls{jfnk}
    method were marginal and often, the power iteration method terminated in
    less time \cite{gill_azmy}. While the choice of a particular language, the
    authors' results do suffer in part due to the choice of an interpreted
    rather than a compiled language.

    \subsubsection{Ragged Core}
      It almost goes without saying that solution time can be expected to be
      strongly related to the number of solving variables. However,
      \citeauthor{qe2paper} include unnecessary nodes in their simulation. The
      simulations presented in the article are in the domain of a rectangular
      prism but standard reactor simulations are performed on a ``ragged core.''
      In order to simplify the geometry description, the authors introduce new
      nodes at the periphery of the problem. Consider the IAEA benchmark. The
      introduction of new nodes does not significantly impact the result as the
      simulation agrees to the benchmark within $4~\units{pcm}$. However, the
      benchmark geometry can be described exactly by 245, $10~\units{cm} \times
      10~\units{cm}$ nodes at each axial elevation and the authors simulate 289
      nodes at each axial elevation. The authors went to great lengths to reduce
      the number of solution variables (see \sref{sec:local_elimination}), but
      approximately 18\% of the nodes solved are absolutely unnecessary for the
      problem.

      It may seem that simulating a ragged core would only further improve the
      relative speedup of the \gls{jfnk} method compared to the \gls{pi}
      method, but such a conclusion is not straightforward. It is known that in
      reactor simulations using the \gls{pi} method, the method may spend many 
      iterations unnecessarily solving for the flux in cells with little or no 
      fissile material \cite{gehinThesis}. Therefore, the reactors are simulated 
      with a ``ragged core'' and the \gls{pi} eigenvector convergence criteria 
      is typically written in terms of the distributed fission rate as
      \begin{equation}
        \label{eq:power_iteration_convergence}
        \epsilon_{\phi} \ge \max_{i} \, \left\lvert 
          \frac{\sum_{g=1}^{G} \nu\Sigma_{f,g,i} \overline{\phi}_{g,i}^{(s)} - 
          \sum_{g=1}^{G} \nu\Sigma_{f,g,i} \overline{\phi}_{g,i}^{(s-1)}}
          {\sum_{g=1}^{G} \nu\Sigma_{f,g,i} \overline{\phi}_{g,i}^{(s)}}
          \right\rvert
      \end{equation}
      where $(s)$ represents the iteration counter. The choice in
      \eref{eq:power_iteration_convergence} ensures nodes without fissile
      material do not inhibit convergence behavior. However, in the work by
      \citeauthor{qe2paper}, the convergence of the residual function
      $\residual(\vx^m)$ is presented. The convergence of the quantities of
      interest, $\lambda$ and $\overline{\phi}$, cannot be inferred. Therefore,
      it is possible that the power iteration method may be spending many
      iterations in these nodes whereas the \gls{jfnk} method would not
      experience such behavior. It is possible that which have been
      artificially introduced and that the calculation of $\lambda$ and
      $\overline{\phi}$ by means of the \gls{pi} method could be terminated much
      earlier.

    \subsubsection{\texorpdfstring{\glsentrylong{ws}}{Wielandt Shift}}
    \label{sec:wielandt_shift}
      The final shortcoming of the comparison of the \gls{jfnk} and power
      iteration methods presented by \citeauthor{qe2paper} is the improper
      consideration of the \gls{ws}. The authors claim that integrating the
      \gls{ws} method into the \gls{jfnk} method will ``improve efficiency
      further'' \cite{qe2paper}. However, both the work by
      \citeauthor{gill_azmy} and the discussion of the \gls{pi} dominance ratio
      in \sref{sec:dominance_ratio} indicate that this is misguided. It is not
      expected that implementing the \gls{ws} will improve the efficiency of the
      \gls{jfnk} method.

      The true shortcoming of the results presented by \citeauthor{qe2paper} is
      that the \gls{pi} method to which the results were compared did not
      implement the \gls{ws}. It is expected that the \gls{ws} can speedup
      results of the power iteration method for these problems by a factor of
      three or more. This would invalidate some of the conclusions by
      \citeauthor{qe2paper} because the \gls{pi} method would then be faster
      than the \gls{jfnk} method for some cases.

      Results by \citeauthor{jfnk_wielandt} indicate that even with the
      \gls{ws}, the \gls{jfnk} method may still provide speedup for the
      multigroup neutron diffusion equation solution to the IAEA benchmark in
      two spatial dimensions compared to the \gls{pi} method
      \cite{jfnk_wielandt}. However, these results may not directly extend to
      the work by \citeauthor{qe2paper} as the solution is presented for the
      \gls{nem} and a heavily modified and specially preconditioned form at
      that.

      Including the \gls{ws} in the \gls{jfnk} method is also shown to be useful
      for preconditioning the Krylov solver with a series of \glspl{pi}. Unlike
      \citeauthor{gill_azmy} and \citeauthor{jfnk_wielandt},
      \citeauthor{qe2paper} do not investigate \gls{pi} preconditioning but
      such a preconditioner warrants investigation.
